\documentclass{article}

\usepackage{amsmath}
\usepackage{amssymb}
\usepackage{indentfirst}
\usepackage{xeCJK}
\usepackage[
    top=1cm,
    bottom=1cm,
    left=1.5cm,
    right=1.5cm,
    headheight=13.6pt,
    a4paper 
]{geometry}
\usepackage{hyperref}
\usepackage{listings}
\hypersetup{
    colorlinks=true,
    linkcolor=blue,
    filecolor=magenta,
    urlcolor=blue,
    citecolor=green
}

% for Chinese fonts, use Simsum
\usepackage{fontspec} 
\setCJKmainfont{SimSun}

% 伪代码宏包
\usepackage{algorithm}
% \usepackage{algpseudocode}
\usepackage{algorithmicx}
\usepackage[noEnd=false,spaceRequire=false,italicComments=true]{algpseudocodex}


\lstset{
    language=Python,         
    basicstyle=\ttfamily,    
    numbers=left,            
    numberstyle=\tiny\color{gray}, 
    frame=single,            
    breaklines=true,         
    keywordstyle=\color{blue}\bfseries, 
    commentstyle=\color{green!60!black},
    stringstyle=\color{red!70!black} 
}

\title{Simple Demo for Pseudocode and Algorithms}
\author{Xiyuan Yang}
\date{\today}

\begin{document}

\maketitle
\section{Introduction}

% use english
Algorithm rules the world!

% use Chinese
我爱算法、我爱算法课。

\section{Pseudocode}

\subsection{Simple Demo}

In this section, we will discuss the code for writing formatted and well-defined pseudocode.

\begin{algorithm}[H]
    \caption{Algorithm for sum}
    \label{alg:sum_algorithm}
    \begin{algorithmic}[1]
        % require: 前置条件
        \Require A list of numbers $L$

        % ensure: 保证结果
        \Ensure the sum of all the elements in the list
        \State sum $\gets 0$
        \LComment{The next two lines
            increment both $x$ and $y$.}
        \For{element $x$ in List $L$}
        \State sum $\gets$ sum $+ x$
        \Comment{add new number x to the sum}
        \EndFor
        \State \textbf{Return} sum
    \end{algorithmic}
\end{algorithm}

Let'e making it more \textbf{complex}!

\begin{algorithm}[H]
    \caption{Binary Search Algorithm}
    \label{alg:binary_search}
    \begin{algorithmic}[1]
        \Require A sorted list of numbers $A$
        \Require A target value $t$ to find
        \Ensure If $t$ exists in list $A$, return its index; otherwise, return $-1$
        \State $low \gets 1$
        \State $high \gets \text{length}(A)$
        \While{$low \le high$}
        \State $mid \gets \text{floor}((low + high) / 2)$
        \Comment{Calculate the middle index}
        \If{$A[mid] = t$}
        \State \textbf{Return} $mid$
        \Comment{Target found, return index}
        \ElsIf{$A[mid] < t$}
        \State $low \gets mid + 1$
        \Comment{Target is in the right half, update low index}
        \Else
        \State $high \gets mid - 1$
        \Comment{Target is in the left half, update high index}
        \EndIf
        \EndWhile
        \State \textbf{Return} $-1$
        \Comment{Loop finished, target not found}
    \end{algorithmic}
\end{algorithm}


\subsection{Detailed Manuals}

Official Docs: \href{https://ctan.math.washington.edu/tex-archive/macros/latex/contrib/algpseudocodex/algpseudocodex.pdf}{algpseudocodex.pdf}.

\subsubsection{Algorithmic Block}

\begin{algorithm}[H]
    \caption{Algorithmic Block}
    \begin{algorithmic}[1]
        \State $E = mc^2$
    \end{algorithmic}
\end{algorithm}

\subsubsection{Several Sentences}

\begin{algorithm}[H]
    \caption{Several Sentences}
    \begin{algorithmic}[1]
        \State It is a state
        \Statex It is a statex
        \Comment{This is a comment}

        \State Hello world
        \State \Call{hello}{$a_1$, $a_2$}
        \Comment{This is a function}
        \State \Return Hello world
    \end{algorithmic}
\end{algorithm}


\subsection{Loops and Other keywords}

\begin{algorithm}[H]
    \caption{Loops and Other keywords}
    \begin{algorithmic}[1]
        \State While Loops
        \While{$x \le 1$}
        \State \Call{Test}{$x$}
        \EndWhile

        \State For Loop
        \For{$n = 1, \dots, 10$}
        \State body
        \EndFor

        % Use single \State for empty line
        \State

        \ForAll{$n \in \{1, \dots, 10\}$}
        \State body
        \EndFor

        \State

        \Loop
        \State body
        \EndLoop

        \State

        % indent is not necessary
        \If{condition}
        \State body
        \ElsIf{condition}
        \State body
        \Else
        \State body
        \EndIf


        \State

        \Procedure{name}{parameters}
        \State body
        \EndProcedure

        \State

        \Function{name}{parameters}
        \State body
        \EndFunction

    \end{algorithmic}
\end{algorithm}


% \subsection{Boxes}

% \begin{algorithm}[H]
%     \begin{algorithmic}[1]
%         \BeginBox[draw=blue, fill=yellow] % 放在这里,框住整个算法
%         \State first line
%         \State second line
%         \State another line
%         \State last line
%         \EndBox
%     \end{algorithmic}
% \end{algorithm}

% \begin{algorithmic}
%     \State first line
%     \State second line with
%     \BoxedString[fill=yellow]{box}
%     \State last line
% \end{algorithmic}

% \begin{algorithm}[H]
%     \begin{algorithmic}[1]
%         \BeginBox % 外部大框
%             \State first line
%             \BeginBox[fill=yellow] % 内部小框
%                 \State second line
%                 \State another line
%             \EndBox
%         \EndBox
%         \BeginBox[draw=blue,dashed]
%             \State last line
%         \EndBox
%     \end{algorithmic}
% \end{algorithm}




\section{Simple Demo for Listings}

\begin{lstlisting}[caption=My first Python code]
# A simple function
def greet(name):
    """
    This function greets the person passed in as a parameter.
    """
    print("Hello," + name + "!")

# Call the function
greet("World")
\end{lstlisting}



\end{document}

